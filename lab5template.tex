% This is the alpha version of lab report template in LaTeX.
% Originally used by Daniel Whisman (enrolled in ECE 486, Fall 2013)
% Edited and redesigned by Y\"un L. Han.
% 2014-10-09
%
% It is recommended to use
%
% TeXLive on Linux, url: http://tug.org/texlive/
% MacTex on Mac OS X, url: http://tug.org/mactex/
% MikTeX on Windows, url: http://miktex.org/

% For more info about how to use LaTeX (not TeX, TeX is a programming
% language, however LaTeX is not), see a quick guide from ECE Thesis wiki,
%
% https://wiki.cites.illinois.edu/wiki/display/ECEThesisReview/LaTeX+Resources
%
% Read the Documentation file ECE LaTeX Guide.pdf down at the bottom
% of the page. As a starter, you only have to learn briefly how to typeset
% equations and insert figures and other graphics. As an experienced user,
% however, you can grab the template and write your report immediately.

\documentclass{article}
\usepackage{graphicx}
\usepackage{caption}
\usepackage{mathtools} % load AMS maths
\usepackage{amsmath} % for math spacing
\usepackage{amssymb} % for math spacing
\usepackage{colortbl} % colour for table
\usepackage[table]{xcolor} % define colours
\usepackage{multirow} % multirow cell in table
\usepackage{array} % table cell, paragraph column aligned in the middle
\usepackage[margin=2.5cm]{geometry} % an easy way to change page layout, Thanks to Brady Salz
\usepackage{siunitx} % use this package to enter SI units, Thanks to Dr. Sean Hubbard
\definecolor{Grey}{gray}{0.85}
\newcommand{\score}{\hfill \underline{\hspace{0.65cm}}\,/} % for score underline
\newcommand\RR{\textsuperscript{\textregistered}~} % for registered mark

\begin{document}
\title{\bf Lab \#5 Report\\{\sc PD Control---Analog Computer and \\ Windows Target}}
\author{Your Name\\ Section AB\emph{X}\\
Your Lab Partner's Name\\
Your TA's name}
\maketitle

\noindent \fbox{\bf \Large Total: \underline{\hspace{0.65cm}}\,/90}

\section{Comparison of Responses  \score 44}
\subsection{Theoretical Performance Criteria \score 8}
% fill out the following four tables
\begin{table}[phtb]\footnotesize 
% http://texblog.org/2014/05/19/coloring-multi-row-tables-in-latex/
\begin{center}
\caption{Theoretical Values according to Figure 5.2 (Lab Manual)}
\label{tbl:lab5_q1_1}
\begin{tabular}{l|m{1.2cm}|m{1.2cm}|m{1.2cm}|m{1.2cm}|m{1.2cm}|m{1.2cm}|m{1.2cm}|m{1.2cm}} \hline \hline
\cellcolor{lightgray} & \multicolumn{2}{c|}{\cellcolor{lightgray}Gains Set 1} & \multicolumn{2}{c|}{\cellcolor{lightgray}Gains Set 2} & \multicolumn{2}{c|}{\cellcolor{lightgray}Gains Set 3} & \multicolumn{2}{c}{\cellcolor{lightgray}Gains Set 4}\\ \cline{2-9}
\multirow{-2}{*}{\cellcolor{lightgray}parameters}& \multicolumn{2}{c|}{$P_1 = 0.15$, $P_2 = 0$}& \multicolumn{2}{c|}{$P_1 = 0.25$, $P_2 = 0.35$}& \multicolumn{2}{c|}{$P_1 = 0.1$, $P_2 = 0.5$}& \multicolumn{2}{c}{$P_1 = ~~~$, $P_2 = ~~~$}\\ \hline
$\zeta$\,[\si{\second}] & \multicolumn{2}{c|}{YOUR VALUE} & \multicolumn{2}{c|}{} & \multicolumn{2}{c|}{} & \multicolumn{2}{c}{} \\ \hline
$\omega_n$\,[\si{\radian\per\second}] & \multicolumn{2}{c|}{} & \multicolumn{2}{c|}{} & \multicolumn{2}{c|}{} & \multicolumn{2}{c}{} \\ \hline
$M_p$\,[\%]& \multicolumn{2}{c|}{} & \multicolumn{2}{c|}{~} & \multicolumn{2}{c|}{~} & \multicolumn{2}{c}{~} \\ \hline
$t_r$\,[\si{\second}] & \multicolumn{2}{c|}{} & \multicolumn{2}{c|}{} & \multicolumn{2}{c|}{} & \multicolumn{2}{c}{} \\ \hline
$t_s$\,[\si{\second}] & \multicolumn{2}{c|}{} & \multicolumn{2}{c|}{} & \multicolumn{2}{c|}{} & \multicolumn{2}{c}{} \\ \hline
\end{tabular}
\end{center}
\end{table}

\subsection{Experimental Performance Criteria \score 12}
\emph{Note: $M_p$, $t_r$, $t_s$ are calculated with respect to the steady state responses, not the reference signals.}
%\subsubsection{Section I---Analog Computer}
\begin{table}[phtb]\footnotesize 
\begin{center}
\caption{Section I---Analog Computer}
\label{tbl:lab5_q1_2}
\begin{tabular}{l|m{1.2cm}|m{1.2cm}|m{1.2cm}|m{1.2cm}|m{1.2cm}|m{1.2cm}|m{1.2cm}|m{1.2cm}} \hline \hline
\cellcolor{lightgray} & \multicolumn{2}{c|}{\cellcolor{lightgray}Gains Set 1} & \multicolumn{2}{c|}{\cellcolor{lightgray}Gains Set 2} & \multicolumn{2}{c|}{\cellcolor{lightgray}Gains Set 3} & \multicolumn{2}{c}{\cellcolor{lightgray}Gains Set 4}\\ \cline{2-9}
\multirow{-2}{*}{\cellcolor{lightgray}parameters}& \multicolumn{2}{c|}{$P_1 = 0.15$, $P_2 = 0$}& \multicolumn{2}{c|}{$P_1 = 0.25$, $P_2 = 0.35$}& \multicolumn{2}{c|}{$P_1 = 0.1$, $P_2 = 0.5$}& \multicolumn{2}{c}{$P_1 = ~~~$, $P_2 = ~~~$}\\ \hline
$M_p$\,[\%]& \multicolumn{2}{c|}{YOUR VALUE} & \multicolumn{2}{c|}{~} & \multicolumn{2}{c|}{~} & \multicolumn{2}{c}{~} \\ \hline
$t_r$\,[\si{\second}] & \multicolumn{2}{c|}{} & \multicolumn{2}{c|}{} & \multicolumn{2}{c|}{} & \multicolumn{2}{c}{} \\ \hline
$t_s$\,[\si{\second}] & \multicolumn{2}{c|}{} & \multicolumn{2}{c|}{} & \multicolumn{2}{c|}{} & \multicolumn{2}{c}{} \\ \hline
\end{tabular}
\end{center}
\end{table}

%\subsubsection{Section II---Windows Target}
\begin{table}[phtb]\footnotesize 
\begin{center}
\caption{Section II---Windows Target}
\label{tbl:lab5_q1_3}
\begin{tabular}{l|m{1.2cm}|m{1.2cm}|m{1.2cm}|m{1.2cm}|m{1.2cm}|m{1.2cm}|m{1.2cm}|m{1.2cm}} \hline \hline
\cellcolor{lightgray} & \multicolumn{2}{c|}{\cellcolor{lightgray}Gains Set 1} & \multicolumn{2}{c|}{\cellcolor{lightgray}Gains Set 2} & \multicolumn{2}{c|}{\cellcolor{lightgray}Gains Set 3} & \multicolumn{2}{c}{\cellcolor{lightgray}Gains Set 4}\\ \cline{2-9}
\multirow{-2}{*}{\cellcolor{lightgray}parameters}& \multicolumn{2}{c|}{$P_1 = 0.15$, $P_2 = 0$}& \multicolumn{2}{c|}{$P_1 = 0.25$, $P_2 = 0.35$}& \multicolumn{2}{c|}{$P_1 = 0.1$, $P_2 = 0.5$}& \multicolumn{2}{c}{$P_1 = ~~~$, $P_2 = ~~~$}\\ \hline
$M_p$\,[\%]& \multicolumn{2}{c|}{YOUR VALUE} & \multicolumn{2}{c|}{~} & \multicolumn{2}{c|}{~} & \multicolumn{2}{c}{~} \\ \hline
$t_r$\,[\si{\second}] & \multicolumn{2}{c|}{} & \multicolumn{2}{c|}{} & \multicolumn{2}{c|}{} & \multicolumn{2}{c}{} \\ \hline
$t_s$\,[\si{\second}] & \multicolumn{2}{c|}{} & \multicolumn{2}{c|}{} & \multicolumn{2}{c|}{} & \multicolumn{2}{c}{} \\ \hline
\end{tabular}
\end{center}
\end{table}

%\subsubsection{Section III---Windows Target with Friction Compensation}
\begin{table}[phtb]\footnotesize 
\begin{center}
\caption{Section III---Windows Target with Friction Compensation}
\label{tbl:lab5_q1_4}
\begin{tabular}{l|m{1.2cm}|m{1.2cm}|m{1.2cm}|m{1.2cm}|m{1.2cm}|m{1.2cm}|m{1.2cm}|m{1.2cm}} \hline \hline
\cellcolor{lightgray} & \multicolumn{2}{c|}{\cellcolor{lightgray}Gains Set 1} & \multicolumn{2}{c|}{\cellcolor{lightgray}Gains Set 2} & \multicolumn{2}{c|}{\cellcolor{lightgray}Gains Set 3} & \multicolumn{2}{c}{\cellcolor{lightgray}Gains Set 4}\\ \cline{2-9}
\multirow{-2}{*}{\cellcolor{lightgray}parameters}& \multicolumn{2}{c|}{$P_1 = 0.15$, $P_2 = 0$}& \multicolumn{2}{c|}{$P_1 = 0.25$, $P_2 = 0.35$}& \multicolumn{2}{c|}{$P_1 = 0.1$, $P_2 = 0.5$}& \multicolumn{2}{c}{$P_1 = ~~~$, $P_2 = ~~~$}\\ \hline
$M_p$\,[\%]& \multicolumn{2}{c|}{YOUR VALUE} & \multicolumn{2}{c|}{~} & \multicolumn{2}{c|}{~} & \multicolumn{2}{c}{~} \\ \hline
$t_r$\,[\si{\second}] & \multicolumn{2}{c|}{} & \multicolumn{2}{c|}{} & \multicolumn{2}{c|}{} & \multicolumn{2}{c}{} \\ \hline
$t_s$\,[\si{\second}] & \multicolumn{2}{c|}{} & \multicolumn{2}{c|}{} & \multicolumn{2}{c|}{} & \multicolumn{2}{c}{} \\ \hline
\end{tabular}
\end{center}
\end{table}


\subsection{Comparison of Sections I and II \score 8}
(Note any differences and characteristic similarities. Should they be the same? If they are different, why do they differ?)
% add your discussion here


\subsection{Comparison of Sections II and III \score 16}
(Note any differences and characteristic similarities. Should they be the same? If they
are different, why do they differ? Also, what are the effects of friction on the response of
the system? How does it affect $M_p$, $t_r$, and $t_s$?)
% add your discussion here

\section{Performance \score 18}
\subsection{Compare the Performance of Your Design to the Specifications  \score 10}
(Did your design meet the specifications given in the prelab ($M_p < 15\%$ and $t_r < 30 \si{\milli\second}$)? If not, can you give some suggestions to improve the performance? For example, which gain could be improved and better?)
% add your discussion here
\subsection{Unmodeled Plant Dynamics \score 8}
(Explain how unmodeled plant might cause problems. Give an example of dynamics that were unmodeled or ignored in the prelab. What problems could these dynamics cause?)
% add your discussion here

\section{Steady State Error \score 16}
% fill out the following table
\subsection{Theoretical and Measured $e_{\rm ss}$ \score 8}
\begin{table}[phtb]\footnotesize 
\begin{center}
\caption{Steady State Error in Each Trial}
\label{tbl:lab5_q3}
\begin{tabular}{c|m{1.2cm}|m{1.2cm}|m{1.2cm}|m{1.2cm}|m{1.2cm}|m{1.2cm}|m{1.2cm}|m{1.2cm}} \hline \hline
\cellcolor{lightgray} & \multicolumn{2}{c|}{\cellcolor{lightgray}Gains Set 1} & \multicolumn{2}{c|}{\cellcolor{lightgray}Gains Set 2} & \multicolumn{2}{c|}{\cellcolor{lightgray}Gains Set 3} & \multicolumn{2}{c}{\cellcolor{lightgray}Gains Set 4}\\ \cline{2-9}
\multirow{-2}{*}{\cellcolor{lightgray}Trials}& \multicolumn{2}{c|}{$P_1 = 0.15$, $P_2 = 0$}& \multicolumn{2}{c|}{$P_1 = 0.25$, $P_2 = 0.35$}& \multicolumn{2}{c|}{$P_1 = 0.1$, $P_2 = 0.5$}& \multicolumn{2}{c}{$P_1 = ~~~$, $P_2 = ~~~$}\\ \hline
Theoretical\,[\%] & \multicolumn{2}{c|}{YOUR VALUE} & \multicolumn{2}{c|}{~} & \multicolumn{2}{c|}{~} & \multicolumn{2}{c}{~} \\ \hline
Section I\,[\%] & \multicolumn{2}{c|}{} & \multicolumn{2}{c|}{} & \multicolumn{2}{c|}{} & \multicolumn{2}{c}{} \\ \hline
Section II\,[\%]& \multicolumn{2}{c|}{} & \multicolumn{2}{c|}{} & \multicolumn{2}{c|}{} & \multicolumn{2}{c}{} \\ \hline
Section III\,[\%] & \multicolumn{2}{c|}{} & \multicolumn{2}{c|}{} & \multicolumn{2}{c|}{} & \multicolumn{2}{c}{} \\ \hline
\end{tabular}
\end{center}
\end{table}

\subsection{Gain Adjustments to Decrease $e_{\rm ss}$ \score 8}
(What gain adjustments helped decrease steady state error? Can you give a general rule for which gain values give the lowest steady state error?)
% add discussion here

\section{Friction Compensation \score 12}
% fill out the following table
\subsection{Friction Values \score 6}
\begin{table}[phtb]\footnotesize
\begin{center}
\caption{Friction Values}
 \label{tbl:lab5_q4}
\begin{tabular}{l|m{2.5cm}|m{2.5cm}|m{2.5cm}} \hline \hline
\cellcolor{lightgray} &\cellcolor{lightgray}  &\cellcolor{lightgray} Lab 5 Section III & \cellcolor{lightgray} Lab 5 Section III \\
\cellcolor{lightgray}&\multirow{-2}{*}{\cellcolor{lightgray}~~~~~~~~~Lab4} & \cellcolor{lightgray}~~~~Full Values &\cellcolor{lightgray} ~Reduced Values \\
\hline
$c^+$\,[\si{\newton\meter}] & & &  \\ \hline
$c^-$\,[\si{\newton\meter}] & & &  \\ \hline
$b^+$\,[\si{\newton\meter\second\per\radian}] & & & \\ \hline
$b^-$\,[\si{\newton\meter\second\per\radian}] & & & \\ \hline
\end{tabular}
\end{center}
\end{table}

\subsection{Discussion \score 6}
(How much did you have to reduce the friction values (in order to get ``stable'' step responses)? How do both results from Lab 5 compare with Lab 4?)
% add your discussion here
\newline \\[3mm]
\noindent {\bf \large Attachments}
\begin{itemize}
\item None necessary % (hooray!)
\end{itemize}
\end{document}

