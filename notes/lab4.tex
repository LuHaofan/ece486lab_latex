% Dan Whisman's template
% DO NOT USE IN 486 LAB
\documentclass{article}
\usepackage{graphicx}
\newcommand{\score}{\hfill \underline{\hspace{1cm}}/}
\title{Lab \#4\\{\sc Introduction to the DC Motor}}
\author{Your Name\\ Section\\
  Partners: Partner's Name\\
  TA: TA's name}
%\renewcommand{\thesection}{\Roman{section}}
\begin{document}
\maketitle
\fbox{\bf \Large Total: \underline{\hspace{1cm}}/90}

\section{Calibration of Tachometer \score 5}
\subsection{Computing the Tachometer Gain \score 3}


\subsection{Experimental Parameters for the Tachometer \score 2}
\begin{center}
\begin{tabular}{rrrrr}
$V_i$ (V) & $\Delta t$ (ms) & $V_{\rm tach}$ (v) & $\omega$ (rad/s) & $K_{\rm tach}$ (V\,s/rad) \\
\hline
5  &  &  &   &  \\
10 &  &  &   &   \\
15 &  &  &   &  
\end{tabular}
\end{center}
The average value of $K_{\rm tach}$ is \,V\,s/rad.

\section{Armature Resistance and Back-EMF \score 10}
\subsection{Measuring the $R_a$ and Back-EMF \score 4}
 
\subsection{Experimental Values \score 2}
\begin{center}
\begin{tabular}{rrrr}
$V_i$ (V) & $I_{\rm ss-a}$ (A) & $V_{\rm tach}$ (V) & $\omega_{\rm ss}$ (rad/s) \\
\hline
5 &   &   &   \\
6 &   &   &   \\
7 &   &   &  \\
8 &   &   &  \\
9 &   &   &  \\
10  &   &   & \\
11  &   &   &  \\
12  &   &   &  \\
-5  &  &  &  \\
-6  &  &  &  \\
-7  &  &  &   \\
-8  &  &  &   \\
-9  &  &  &   \\
-10 &  &  &   \\
-11 &  &  &   \\
-12 &  &  & 
\end{tabular}
\end{center}
\subsection{Experimental Parameters \score 4}
\begin{center}
\begin{tabular}{cc}
Parameter & Value \\
\hline
$R_a\,(\Omega)$ &   \\
$K_V$ (V\,s/rad) &  
\end{tabular}
\end{center}

\section{Friction Coefficients \score 10}
\subsection{Measuring the Coefficients \score 6}


The plots of frictional torque vs. angular velocity are shown in Figures~\ref{fig:fp1} and~\ref{fig:fp2}.

\begin{figure}[hptb]
\centering
%\includegraphics[width=0.9\linewidth]{fric_plot1.eps}
\caption{Frictional torque vs. angular velocity for positive $\omega$.}
\label{fig:fp1}
\end{figure}

\begin{figure}[hptb]
\centering
%\includegraphics[width=0.9\linewidth]{fric_plot2.eps}
\caption{Frictional torque vs. angular velocity for negative $\omega$.}
\label{fig:fp2}
\end{figure}

\subsection{Experimental Values \score 4}
\begin{center}
\begin{tabular}{lr|lr}
\multicolumn{2}{c}{Viscous Coefficient (Nms/rad)} & \multicolumn{2}{c}{Coulomb Coefficient (Nm)} \\
\hline
$b^+$ &   &   &   \\
$b^-$ &   &   &  
\end{tabular}
\end{center}

\section{Armature Inductance \score 20}
\subsection{Procedure for Measuring $R_s$ and $L_a$ \score 14}


\begin{figure}[htbp]
\centering
%\includegraphics[width=0.9\linewidth]{La_plot.eps}
\caption{Plot of $\log(V)$ vs. $t$. }
\label{fig:La_plot}
\end{figure}


\subsection{Experimental Parameters \score 6}
$R_s$ was measured to be   .
\begin{center}
\begin{tabular}{crr}
Trial & $\tau_e$\,(ms) & $L_a$\,(mH) \\
\hline
1 &  &  \\
2 &  &  \\
3 &  &  \\
4 &  &  \\
5 &  &  \\
6 &  &  \\
Average &  & 
\end{tabular}
\end{center}

\section{Rotor Moment of Inertia \score 18}
\subsection{Procedure for Measuring $J$ \score 14}

\begin{figure}[htbp]
\centering
%\includegraphics[width=0.9\linewidth]{iner_plot.eps}
\caption{Plot of $\log(\omega)$ vs. $t$.  The line of best fit is $\log(\omega) = -0.403t + 6.00$. }
\label{fig:iner_plot}
\end{figure}


\subsection{Experimental Values for $J$ \score 4}
\begin{center}
\begin{tabular}{cr}
Trial & Inertia ($10^{-4}$kg\,m$^2$) \\
\hline
1 &  \\
2 &  \\
3 &  \\
4 &  \\
5 &  \\
6 &  \\
Average & 
\end{tabular}
\end{center}

\section{Conservation of Energy \score 12}


\section{System Transfer Function \score 10}
\subsection{Transfer Function $\Omega(s)/V_i(s)$ \score 5}
The transfer function is given by
\begin{eqnarray*}
\frac{\Omega(s)}{V_i(s)} &=& \\
&=& 
\end{eqnarray*}
The poles are located at $s = $ .

\subsection{Transfer Function $\Omega_{\rm approx}(s)/V_i(s)$ \score 5}
The first order approximation is given by
\begin{eqnarray*}
\frac{\Omega_{\rm approx}(s)}{V_i(s)} &=& \\
&=&.
\end{eqnarray*}
The pole is at $s=-$ .
\section{Steady-State Response of Non-Linear System \score 5}

\end{document}
