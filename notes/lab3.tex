% Dan Whisman's template
% DO NOT USE IN 486 LAB
\documentclass{article}
\usepackage{graphicx}
\newcommand{\score}{\hfill \underline{\hspace{1cm}}/}
\title{Lab \#3\\{\sc Digital Simulation}}
\author{Your Name\\ Section\\
  Partners: Partner's Name\\
  TA: TA's name}
%\renewcommand{\thesection}{\Roman{section}}
\begin{document}
\maketitle
\fbox{\bf \Large Total: \underline{\hspace{1cm}}/45}
\section{{\sc Results}\score 15}
\subsection{Plots \score 6}
See Figures~\ref{ref_resp} and~\ref{dist_resp} for the responses to a unit step as the reference and disturbance, respectively.

\begin{figure}[htbp]
\centering
%\includegraphics[height=0.4\textheight]{ref_resp.pdf}
\caption{Reference response of three controllers.}
\label{ref_resp}
\end{figure}

\begin{figure}[htbp]
\centering
%\includegraphics[height=0.4\textheight]{dist_resp.pdf}
\caption{Disturbance response of three controllers.}
\label{dist_resp}
\end{figure}

\subsection{Step Response to $\omega_r$ \score 6}
%\input{data.tex}

\subsection{Comparison \score 3}


\section{{\sc Deriving $e_{ss}$ Components} \score 12}

\section{{\sc $\zeta$ and $\omega_n$ of Controller Three} \score 18}


\begin{figure}[htbp]
\centering
%\includegraphics[height=0.4\textheight]{report_plot.pdf}
\caption{Pole locations for $0<k_d<0.05$.}
\label{poles}
\end{figure}

\end{document}
