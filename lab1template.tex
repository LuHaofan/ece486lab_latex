% LaTeX template for ECE 486 Lab; this template should be used with "report"
% section in the lab manual

% by Y\"un Han
% 2017-06-06

% A lot of changes have happened during the last four years. So a major overhaul
% is needed.

% -------------------
% Editor + TeX engine
% -------------------
% There are many online TeX services which require no local installation of TeX
% on your computer. Over the years, I have seen students use
% https://www.overleaf.com https://www.sharelatex.com among others although I
% have no direct experience of any of them. If you do not want any hassle of
% having a copy of TeX on your machine, or you don't mind privacy policies of
% those online services, you can use the online TeX system as well. They will
% serve you well, at least for our lab report.

% If you are really interested in beautiful print, and efficient editing, you
% can explore other options. For example, I use local TeX installations on my
% Linux, Mac and Windows machines. See the original preamble regarding how to
% choose a proper distribution for different platforms.

% As for editors, I use emacs + AUCTeX. You can enable major modes to handle
% *.tex files and minor modes to streamline your typesetting. For example, most
% people will use PDF mode rather than output files in *.ps or *.dvi these
% days; LaTeX-math-mode will speed up entering common math symbols and Greek
% letters; auto-complete will do a similar thing; flyspell mode will give you
% spell check etc.

% As for setting up emacs itself, you can check out Jess Hamrick's github repo
% https://github.com/jhamrick/emacs. I only enabled auctex, auto-complete,
% color-them-solarized, helm, helm-descbinds, and nyan-mode. To get these these
% packages installed, you can use el-get package management for emacs. 

% For ECE 486 however, portable set-up is already available on lab computers
% N:\labs\ECE486\ece486_Emacs_LaTeX
% Configuration files are also available from this repo
% https://github.com/yunlhan/ece486lab_latex/tree/master/emacs_latex

% ------------
% LaTeX how-to 
% ------------
% I use the following
% https://en.wikibooks.org/wiki/LaTeX
% https://en.wikibooks.org/wiki/LaTeX/Tips_and_Tricks
% https://tex.stackexchange.com

% ---------------
% Working example
% ---------------
% I wrote a book (127 pages) based on my lectures notes. You can scan the book
% and you may find it helpful when you use LaTeX. For example, how to handle
% cases in equations, how to selectively label questions, advanced graphics,
% references, just name a few. The book is in available in another repo
% https://github.com/yunlhan/ENT40days

% -------------
% Writing guide
% -------------
% You may find the Guardian style guide useful,
% https://www.theguardian.com/guardian-observer-style-guide-a Especially, their
% guide on "quotation marks" and "commas".

% -----------------------------------------------------------------------------

% original preamble
%
% This is the alpha version of lab report template in LaTeX. Inspired by the
% effort of Daniel Whisman (enrolled in ECE 486, Fall 2013)
% written by Y\"un L. Han.
% 2014-09-07

% It is recommended to use 
%
% TeXLive on Linux, url: http://tug.org/texlive/
% MacTex on Mac OS X, url: http://tug.org/mactex/
% MikTeX on Windows, url: http://miktex.org/

% For more info about how to use LaTeX (not TeX, TeX is a programming
% language, however LaTeX is not), see a quick guide from ECE Thesis wiki,

% https://wiki.cites.illinois.edu/wiki/display/ECEThesisReview/LaTeX+Resources

% Read the Documentation file ECE LaTeX Guide.pdf down at the bottom 
% of the page. As a starter, you only have to learn briefly how to typeset 
% equations and insert figures and other graphics. As an experienced user,
% however, you can grab the template and write your report immediately.

\documentclass{article}
% \usepackage{graphicx}
% \usepackage{caption}
\usepackage{mathtools} % load AMS maths
\usepackage{amsmath}  % for math spacing
\usepackage{amssymb}  % for math spacing
\usepackage[margin=2.5cm]{geometry} % an easy way to change page layout, Thanks to Brady Salz 
\newcommand{\score}{\hfill \underline{\hspace{0.65cm}}\,/} % for score underline
\newcommand\RR{\textsuperscript{\textregistered}~} % for registered mark
\begin{document}
% consider aligning your names like the following
% ----
% ************Lab X Report 
%
% ************by YOUR NAME
% ******Lab Partner: HIS/HER NAME
% ******Lab      TA: Y\"un Han
% ----
\title{\bf Lab \#1 Report\\{\sc Simulation Using \\the Analog Computer}}
\author{Your Name\\ Section AB\emph{X}\\
  Your Lab Partner's Name\\
  Your TA's name}
\maketitle
\noindent \fbox{\bf \Large Total: \underline{\hspace{0.65cm}}\,/40}
\section*{Question 1 \score 15}
\subsection*{Theoretical and Experimental Results \score 5}
\begin{table}[phtb]
  \begin{center}
    \caption{Theoretical and Experimental Results}
     \label{tbl:lab1_q1}
    \begin{tabular}{c|rr|rr|rr} \hline \hline
      & \multicolumn{2}{c}{$M_p$ (\%)} & \multicolumn{2}{c}{$t_r$ (s)} & \multicolumn{2}{c}{$t_s$ (s)} \\ \cline{2-7} % horizontal line spanning across specified columns
      $\zeta$ & Theory & Experiment & Theory & Experiment & Theory & Experiment \\
      \hline
      2.0 & &&&&&\\ 
      1.5 & &&&&&\\
      1.0 & &&&&&\\
      0.8 & &&&&&\\
      0.7 & &&&&&\\
      0.5 & &&&&&\\
      0.3 & &&&&&\\
      0.2 & &&&&& \\ \hline 
    \end{tabular} 
    \end{center}
\end{table}
\subsection*{Comparison of Theoretical/Experimental Results \score 5}
% add your discussion here 

\subsection*{Discussion of Variation of $\zeta$ with $M_p$, $t_r$, and $t_s$ \score 5}
% add your discussion here
% for example
(As $\zeta$ decreases, how does $M_p$ change? As $\zeta$ decreases, how about $t_r$ and $t_s$?)

\section*{Question 2 \score 15}

\subsection*{Effect of $\zeta$ on Pole Locations \score 5}
(Solve for poles in terms of $\zeta$ and sketch a plot of trajectory of pole locations [you can use {\sc Matlab\RR}] when $\zeta$ varies.)
% add your discussion here

\subsection*{Effect of Pole Locations on $M_p$, $t_r$, and $t_s$ for an Underdamped System \score 5}
(What is the value of $\zeta$ when a system is \emph{underdamped}? As $\zeta$ increases, what happens to $M_p$, $t_r$, and $t_s$?)
% add your discussion here

\subsection*{Effect of Pole Locations on $M_p$, $t_r$, and $t_s$ for an Overdamped/Critically Damped System \score 5}
(What is the value of $\zeta$ when a system is \emph{overdamped}? \emph{Critically damped}? And as $\zeta$ increases, what happens to $M_p$, $t_r$, and $t_s$?)
% add your discussion here

\section*{Question 3 \score 10}

\subsection*{Comparison of $2^{\rm nd}$ Order System with $1^{\rm st}$ Order System with Dominant Pole \score 6}
(What are the similarities/differences between the response of an overdamped $2^{\rm nd}$ order system to the response of a $1^{\rm st}$ order system with the \emph{dominant} (less negative, closer to the origin) pole of the $2^{\rm nd}$ order's poles?)
% add your discussion here

\subsection*{Effect of $\zeta$ on Accuracy of Approximation \score 4}
(What is the effect of magnitude of $\zeta$ on the accuracy of the approximations? Also include your graphs.)
% add your discussion here
\newline \\[10mm]
\noindent {\bf \large Attachments}
\begin{itemize}
\item Plots obtained during lab
\item Sample response with relevant points for calculating $M_p$ , $t_s$ and $t_r$ marked
\item Step responses comparing $2^{\rm nd}$ order systems and $1^{\rm st}$ order approximations
\item {\sc Matlab} code
\end{itemize}

\end{document}
