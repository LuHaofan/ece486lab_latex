% This is the alpha version of lab report template in LaTeX.
% Originally used by Daniel Whisman (enrolled in ECE 486, Fall 2013)
% Edited and redesigned by Y\"un L. Han.
% 2014-11-02
%
% It is recommended to use
%
% TeXLive on Linux, url: http://tug.org/texlive/
% MacTex on Mac OS X, url: http://tug.org/mactex/
% MikTeX on Windows, url: http://miktex.org/

% For more info about how to use LaTeX (not TeX, TeX is a programming
% language, however LaTeX is not), see a quick guide from ECE Thesis wiki,
%
% https://wiki.cites.illinois.edu/wiki/display/ECEThesisReview/LaTeX+Resources
%
% Read the Documentation file ECE LaTeX Guide.pdf down at the bottom
% of the page. As a starter, you only have to learn briefly how to typeset
% equations and insert figures and other graphics. As an experienced user,
% however, you can grab the template and write your report immediately.

\documentclass{article}
\usepackage{graphicx}
\usepackage{caption}
\usepackage{mathtools} % load AMS maths
\usepackage{amsmath} % for math spacing
\usepackage{amssymb} % for math spacing
\usepackage{colortbl} % colour for table
\usepackage[table]{xcolor} % define colours
\usepackage{multirow} % multirow cell in table
\usepackage{array} % table cell, paragraph column aligned in the middle
\usepackage[margin=2.5cm]{geometry} % an easy way to change page layout, Thanks to Brady Salz
\usepackage{siunitx} % use this package to enter SI units, Thanks to Dr. Sean Hubbard
% \usepackage{framed} % use boxed environment
\usepackage{hhline} % fix \cline overwritten problem
\usepackage{colortbl} % colour interacts with \hhline
% \definecolor{Grey}{gray}{0.85}
\newcommand{\score}{\hfill \underline{\hspace{0.65cm}}\,/} % for score underline
\newcommand\RR{\textsuperscript{\textregistered}~} % for registered mark

\begin{document}
\title{\bf Lab \#6 Report\\{\sc Lead Controller Design}}
\author{Your Name\\ Section AB\emph{X}\\
Your Lab Partner's Name\\
Your TA's name}
\maketitle

\noindent \fbox{\bf \Large Total: \underline{\hspace{0.65cm}}\,/90}

\section{Effects of Saturation Block \score 10}
(Discuss the effect of the saturation block in the simulations. Overlay plots for the saturated and plots for the non-saturated cases.)
% add your discussion here

% uncomment the following to insert your own figures
% \begin{figure}[phtb]
% \centering
% \includegraphics[height=0.88\textheight]{OVERLAID_05V}
% \caption{Step responses---with and without saturation block for $V_{\rm ss}=0.5\si{\volt}$.}
% \label{fig:lab6_q1_v05}
% \end{figure}

% \begin{figure}[phtb]
% \centering
% \includegraphics[height=0.88\textheight]{OVERLAID_5V}
% \caption{Step responses---with and without saturation block for $V_{\rm ss}=5\si{\volt}$.}
% \label{fig:lab6_q1_v5}
% \end{figure}

% \begin{figure}[phtb]
% \centering
% \includegraphics[height=0.88\textheight]{OVERLAID_50V}
% \caption{Step responses---with and without saturation block for $V_{\rm ss}=50\si{\volt}$.}
% \label{fig:lab6_q1_v50}
% \end{figure}

\section{Motor DC Gain \score 10}
(Compute the transfer function of the system from the data set obtained from the DSA[system identification]. Then compute the DC gain $K$ of the motor model. You need to include how you solve for $K$. Also compare the computed value with the value obtained in Lab 4 and Prelab 5-a. \emph{Note: First, you need to rewrite the transfer function from the DSA in radians. Remember that the DC gain K obtained from the DSA includes the gains of the tachometer, amplifier and compensator. Hence you need to solve only for $K$.})

\begin{align*}
   K_{\rm DSA} & =\mbox{\rm {YOUR VALUE}} \\ 
z_{\rm} \mbox{[\si{\hertz}]} = \mbox{\rm YOUR VALUE},~p_{1\rm} \mbox{[\si{\hertz}]} & =\mbox{\rm YOUR VALUE},~p_{2\rm} \mbox{[\si{\hertz}]} =\mbox{\rm YOUR VALUE}
\end{align*}

\noindent The transfer function is
\begin{align*}
\frac{V_{\rm tach}(s)}{V_{\rm in}(s)} = 
\frac{K_{\rm DSA}\left( \frac{s[\si{\radian\per\second}]}{2\pi}-z[\si{\hertz}] \right)}{\left(\frac{s[\si{\radian\per\second}]}{2\pi}-p[\si{\hertz}]\right)\left(\frac{s[\si{\radian\per\second}]}{2\pi}-p[\si{\hertz}]\right)} = \mbox{YOUR COMPUTATION}
\end{align*}

\begin{center}
\framebox{$K_{\rm lab4} = ~~~~ $, $K_{\rm lab6} = ~~~~$}
\end{center}
% add your comparison here

\section{Time Domain Specifications of Real/Simulated Responses of the Low/High DC Gain Compensators \score 30}
\subsection{Table of $M_p$, $t_r$, $t_s$, and $e_{\rm ss}$ \score 10}
(Fill out Table~\ref{tbl:lab6_q3_1} below.)

\begin{table}[phbt]\footnotesize 
% http://texblog.org/2014/05/19/coloring-multi-row-tables-in-latex/
% http://tex.stackexchange.com/questions/65582/cellcolor-overwrites-table-lines
% http://tex.stackexchange.com/questions/65604/hhline-adds-white-line-where-no-line-is-intended
% http://tex.stackexchange.com/questions/65231/cellcolor-overwrites-partial-horizontal-lines-cline
\begin{center}
\caption{Real and Simulated Responses for Low and High DC Gain Compensators}
\label{tbl:lab6_q3_1}
\begin{tabular}{l|*{7}{m{1.2cm}|}m{1.2cm}} \hline \hline
\cellcolor{lightgray} & \multicolumn{4}{c|}{\cellcolor{lightgray}Low DC Gain Lead Compensator}  & \multicolumn{4}{c}{\cellcolor{lightgray}High DC Gain Lead Compensator}\\ \hhline{>{\arrayrulecolor{lightgray}}-*{8}{>{\arrayrulecolor{black}}-}} % if \cline used, the \hline will be overwritten by \cellcolor, see the above examples
\multirow{-2}{*}{\cellcolor{lightgray}}& \cellcolor{lightgray} $M_p$\,[\%] & \cellcolor{lightgray} $t_r$\,[\si{\milli\second}] & \cellcolor{lightgray} $t_s$\,[\si{\milli\second}] & \cellcolor{lightgray} $e_{\rm ss}$\,[\si{\volt}]$^a$ & \cellcolor{lightgray} $M_p$\,[\%] & \cellcolor{lightgray} $t_r$\,[\si{\milli\second}] & \cellcolor{lightgray} $t_s$\,[\si{\milli\second}] & \cellcolor{lightgray} $e_{\rm ss}$\,[\si{\volt}] \\ \hline
\cellcolor{lightgray} &  & & & & & & & \\
\multirow{-2}{*}{\cellcolor{lightgray}Actual$^b$} & \multirow{-2}{*}{VALUE}& \multirow{-2}{*}{VALUE}& \multirow{-2}{*}{VALUE}& \multirow{-2}{*}{VALUE}& \multirow{-2}{*}{VALUE}& \multirow{-2}{*}{VALUE}& \multirow{-2}{*}{VALUE}& \multirow{-2}{*}{VALUE}\\ \hline
\cellcolor{lightgray} &  & & & & & & & \\
\multirow{-2}{*}{\cellcolor{lightgray}Simulated Lab 4$^c$} & \multirow{-2}{*}{~}& \multirow{-2}{*}{~}& \multirow{-2}{*}{~}& \multirow{-2}{*}{~}& \multirow{-2}{*}{~}& \multirow{-2}{*}{~}& \multirow{-2}{*}{~}& \multirow{-2}{*}{~}\\ \hline
\cellcolor{lightgray} &  & & & & & & & \\
\multirow{-2}{*}{\cellcolor{lightgray}Simulated Lab 6$^d$} & \multirow{-2}{*}{~}& \multirow{-2}{*}{~}& \multirow{-2}{*}{~}& \multirow{-2}{*}{~}& \multirow{-2}{*}{~}& \multirow{-2}{*}{~}& \multirow{-2}{*}{~}& \multirow{-2}{*}{~}\\ \hline
\end{tabular}
\end{center}
\footnotesize{$^a$The unit is [\si{\volt}], not [\%];\\
$^b$ Values calculated from data set obtained in Sections II and III of Lab 6;\\
$^c$ Values calculated from data set by simulating the closed loop system with both lead controllers and the parameters found in Lab 4;\\
$^d$ Values calculated from data set by simulating the closed loop system with both lead controllers and the parameters found in Lab 6 from the DSA.
} 
\end{table}

\subsection{Discussion of Results in Table~\ref{tbl:lab6_q3_1} \score 10}
(Compare the specs in Table~\ref{tbl:lab6_q3_1} for low DC gain and high DC gain compensators and discuss the possible differences.)
% add your discussion here

\subsection{Meeting the Specifications \score 10}
(Fill out Table~\ref{tbl:lab6_q3_2} below. Did your design meet the specification---$M_p \leqslant 15\%$, $t_r \leqslant 30\si{\milli\second}$?)
\begin{table}[phbt] 
\begin{center}
\caption{Overshoot and Rise Time for Lead Compensators and PD Controller}
\label{tbl:lab6_q3_2}
\begin{tabular}{c|*{1}{m{1.2cm}|}m{1.2cm}} \hline \hline
\cellcolor{lightgray} Experimental Values & \cellcolor{lightgray} $M_p$\,[\%] & \cellcolor{lightgray} $t_r$\,[\si{\milli\second}] \\ \hline
\cellcolor{lightgray} Low DC Gain Lead Compensator & & \\ \hline
\cellcolor{lightgray} High DC Gain Lead Compensator & & \\ \hline
\cellcolor{lightgray} PD Controller$^e$ & & \\ \hline 
\end{tabular}
\end{center}
\footnotesize{$^e$Use the PD controller from Lab 5 with gains designed in Prelab 5-c.}
\end{table}
% add your discussion here


\section{Bode Plots \score 40}
\subsection{Bode Plots of $\displaystyle \frac{V_{\rm tach}(s)}{V_{\rm in}(s)}$ \score 10}
\label{sec:tach_bode}
(Plot the Bode plots (Magnitude and Phase) in dB and degrees respectively for the empirical data saved in \texttt{fresp.m}. And obtain Bode plots again using the transfer function estimated by the DSA (part II of this report). Overlay both plots. Use \texttt{semilogx} to scale the $x$-axis (frequency). {\bf They should match well}.)
% insert your overlaid plots 

% uncomment the following to insert your own figure
% \begin{figure}[phtb]
% \centering
% \includegraphics[height=0.88\textheight]{VTACH_VIN}
% \caption{Overlaid Bode plots of open loop transfer function $V_{\rm tach}(s)/V_{\rm in}(s)$.}
% \label{fig:lab6_q4_1}
% \end{figure}

\subsection{Bode Plots of $\displaystyle \frac{V_{\theta}(s)}{V_{\rm in}(s)}$ \score 15}
\label{sec:pot_bode}
(Use the relation between Equations 6.2 and 6.3 (on page 57) of Lab Manual to graph the Bode plots for the transfer function $\displaystyle \frac{V_{\theta}(s)}{V_{\rm in}(s)}$ (in dB and degrees) using the parameters identified by the DSA. Use \texttt{bode} command in {\sc Matlab}\RR to obtain magnitude and phase. Then overlay both plots. Follow hints in Lab Manual at the end of Lab 6 and overlay the \emph{bode plot with the data} in \texttt{fresp.m} (after computing $\displaystyle \frac{V_{\theta}(s)}{V_{\rm in}(s)}$). {\bf They should match well}. Find the crossover frequency ($\omega_c$) and phase margin (PM). Do they meet the design specifications from Prelab 6-d?)
% insert your overlaid plots

% uncomment the following to insert your own figure
% \begin{figure}[phtb]
% \centering
% \includegraphics[height=0.88\textheight]{VTHETA_VIN}
% \caption{Overlaid Bode plots of open loop transfer function $V_{\theta}(s)/V_{\rm in}(s)$.}
% \label{fig:lab6_q4_2}
% \end{figure}

% add your discussion here
\begin{center}
\framebox{$\omega_c = ~~~~ \si{\radian\per\second}$, $\mathrm{PM} = ~~~~ \si{\degree}$ from Data Fit in DSA}\\[2pt]
\framebox{$\omega_c = ~~~~ \si{\radian\per\second}$, $\mathrm{PM} = ~~~~ \si{\degree}$ from data in \texttt{fresp.m}}\\[2pt]
\framebox{$\omega_c = ~~~~ \si{\radian\per\second}$, $\mathrm{PM} = ~~~~ \si{\degree}$ from specification in Prelab 6-d}
\end{center}

\subsection{Bode Plots of Closed-Loop $\displaystyle \frac{V_{\theta}(s)}{V_{\rm in}(s)}$ \score 15}
\label{sec:pot_bode_cl}
(Use the transfer function from the DSA fit (parameters identified by the DSA) and compute the closed-loop transfer function (assume negative unit feedback). Generate a Bode plot using {\sc Matlab}\RR. Also generate a bode plot for closed-loop transfer function $\displaystyle \frac{V_{\theta}(s)}{V_{\rm in}(s)}$ \emph{based on data set} \texttt{fresp.m}. Then overlay both Bode plots. {\bf They should match well}. What is the closed-loop bandwidth? Is it reasonable? Give the reason why it is reasonable or discuss the possible reasons why it is not.)
% insert your overlaid plots
% uncomment the following to insert your own figure
% \begin{figure}[phtb]
% \centering
% \includegraphics[height=0.88\textheight]{VTHETA_VIN_CL}
% \caption{Overlaid Bode plots of closed-loop transfer function $V_{\theta}(s)/V_{\rm in}(s)$ with unit feedback.}
% \label{fig:lab6_q4_3}
% \end{figure}
% add discussion here
\newline \\[3mm]
\noindent {\bf \large Attachments}
\begin{itemize}
\item Overlaid responses from saturated and non-saturated cases for $V_{\rm in} =0.5\si{\volt}$, $V_{\rm in} =5\si{\volt}$ and $V_{\rm in} =50\si{\volt}$ (one plot or three subplots)
\item Overlaid Bode Plots for Section~\ref{sec:tach_bode} for $ \frac{V_{\rm tach}(s)}{V_{\rm in}(s)}$ (one plot)
\item Overlaid Bode Plots for Section~\ref{sec:pot_bode} for $ \frac{V_{\theta}(s)}{V_{\rm in}(s)}$ (one plot)
\item Overlaid Bode Plots for Section~\ref{sec:pot_bode_cl} for closed-loop $ \frac{V_{\theta}(s)}{V_{\rm in}(s)}$ (one plot)
\end{itemize} 

% YOU'RE DONE!!! 
% ENJOY POST ECE 486 LIFE!!!
% BEST OF LUCK!!! :)

\end{document}
