% This is the alpha version of lab report template in LaTeX.
% Originally used by Daniel Whisman (enrolled in ECE 486, Fall 2013)
% Edited and redesigned by Y\"un L. Han.
% 2014-09-12

% It is recommended to use 
%
% TeXLive under Linux, url: http://tug.org/texlive/
% MacTex under Mac OS X, url: http://tug.org/mactex/
% MikTeX under Windows, url: http://miktex.org/

% For more info about how to use LaTeX (not TeX, TeX is a programming
% language, however LaTeX is not), see a quick guide from ECE Thesis wiki,

% https://wiki.cites.illinois.edu/wiki/display/ECEThesisReview/LaTeX+Resources

% Read the Documentation file ECE LaTeX Guide.pdf down at the bottom 
% of the page. As a starter, you only have to learn briefly how to typeset 
% equations and insert figures and other graphics. As an experienced user,
% however, you can grab the template and write your report immediately.

\documentclass{article}
\usepackage{graphicx}
\usepackage{caption}
\usepackage{mathtools} % load AMS maths
\usepackage{amsmath}  % for math spacing
\usepackage{amssymb}  % for math spacing
\usepackage{colortbl} % colour for table
\usepackage{xcolor} % define colours
\usepackage[margin=2.5cm]{geometry} % an easy way to change page layout, Thanks to Brady Salz 
\usepackage{multirow}
\definecolor{Grey}{gray}{0.85}

\newcommand{\score}{\hfill \underline{\hspace{0.65cm}}\,/} % for score underline
\newcommand\RR{\textsuperscript{\textregistered}~} % for registered mark
\begin{document}
\title{\bf Lab \#2 Report\\{\sc Digital Simulation}}
\author{Your Name\\ Section AB\emph{X}\\
  Your Lab Partner's Name\\
  Your TA's name}
\maketitle
\noindent \fbox{\bf \Large Total: \underline{\hspace{0.65cm}}\,/50}
\section{{\sc State Space Model of} $H_1(s)$\score 8}
(Compare the plots of \verb|y_dot| and \verb|y| obtained in Part 1 of the lab with the plots previously made for the prelab. Why are they identical? Attach plots---if your prelab plot was wrong, fix it and attach the corrected plot.)
% add your discussion here

% the following are templates for two FIGURES
\noindent The plots of $y$ and $\dot{y}$ from the prelab and lab are shown in Figures~\ref{fig:prelab2} and~\ref{fig:lab2}.

% uncomment the following to insert your OWN figures
% \begin{figure}[htbp]
% \centering
% \includegraphics[width=0.88\linewidth]{prelab/prelab2_step.png}
% \caption{Plots of $y$ and $\dot{y}$, from the prelab, for a step input.}
% \label{fig:prelab2}
% \end{figure}

% \begin{figure}[htbp]
% \centering
% \includegraphics[width=0.88\linewidth]{lab2_step.png}
% \caption{Plots of $y$ and $\dot{y}$, from the lab, for a step input.}
% \label{fig:lab2}
% \end{figure}

\section{{\sc Effects of an extra Zero}\score 22}
% attach the figure in lab 2 of step responses after adding a zero
The step responses after adding an extra zero are shown in Figure~\ref{fig:effect_zeros}.
% \begin{figure}[htbp]
% \centering
% \includegraphics[width=0.84\linewidth]{zero_added.png}
% \caption{Effects of an extra zero on the step response.}
% \label{fig:effect_zeros}
% \end{figure}


\subsection{Effects of a Zero on $M_p$, $t_r$, and $t_s$\score 2}
Fill the table of specs of time domain responses.
% also attach a sample figure showing how you calculate the specs of the step responses [using your lab 0 script]

\begin{table}[phtb]\footnotesize \label{tbl:lab2_q2}
\begin{center}
\caption{Effects of Zero}
\begin{tabular}{c|c|c|c|c|c|c} \hline \hline
\rowcolor{Grey} & No zero & $H_2(s)$ zero &  $H_2(s)$ zero &  $H_2(s)$ zero  &  $H_2(s)$ zero &  $H_2(s)$ zero \\
\rowcolor{Grey} Specs & $H_1(s)$ & at $s = -30$ & at $s = -3$ &  at $s = -1.5$ &   at $s = 1.8$ &   at $s = 18$ \\ \hline
$M_p$ &  &  &  &  &  &  \\ \hline
$t_r$ &  &  &  &  &  &  \\ \hline
$t_s$ &  &  &  &  &  & \\ \hline
\end{tabular}
\end{center}
\end{table}

\subsection{Discuss the Effects of a LHP Zero\score 4}
(Explain how $M_p$ , $t_r$ , and $t_s$ are affected by the zero location. When can the zero be ignored?)
% add your discussion here

\subsection{Effects of a Non-minimum Phase Zero\score 2}
(What is unique in this situation?)
% add your discussion here

\subsection{Decomposition of $H_2(s)$\score 14}
(Take $H_2(s)$, set $\zeta$ to the value found in the prelab, separate the numerator into two terms so that $H_2(s)$ is a sum of 2 fractions. Discuss how this decomposition helps to explain the effect of the zero location. In particular, discuss what each term represents. Also discuss $\alpha$'s effect. Which term dominates as $\alpha$ approaches 0? As $\alpha$approaches $\infty$?  What happens when $\alpha$ is negative?)

\begin{align*}
H_2(s) &=\frac{25 \left( 1+\frac{s}{\alpha \zeta} \right)}{s^2+10\zeta s +25}\\
       &= \mbox{\sc insert fraction 1} + \mbox{\sc insert fraction 2}
\end{align*}
% complete the above decomposition and add your discussion here

\section{{\sc Effects of an extra Pole}\score 20}
% attach the figure in lab 2 of step responses after adding a pole
The step responses after adding an extra pole are shown in Figure~\ref{fig:effect_pole}.
% uncomment the following to insert your OWN figure
% \begin{figure}[htbp]
% \centering
% \includegraphics[width=0.88\linewidth]{pole_added.png}
% \caption{Effects of an extra pole on the step response.}
% \label{fig:effect_poles}
% \end{figure}

\subsection{Effects of a Pole on $M_p$, $t_r$, and $t_s$\score 2}
(Fill the table of specs of time domain responses.)
\begin{table}[phtb] \label{tbl:lab2_q3}
\begin{center}
\caption{Effects of Pole}
\begin{tabular}{c|c|c|c|c} \hline \hline
\rowcolor{Grey} & No pole & $H_2(s)$ pole & $H_2(s)$ pole & $H_2(s)$ pole \\
\rowcolor{Grey} Specs &  $H_1(s)$ &  at $s = -30$ &  at $s = -3$ & at $s = -1.5$ \\ \hline   
$M_p$ &  &  &  &  \\ \hline
$t_r$ &  &  &  &  \\ \hline
$t_s$ &  &  &  &  \\ \hline
\end{tabular}
\end{center}
\end{table}

\subsection{Discuss the Effects of an Extra Pole\score 4}
(Explain how $M_p$ , $t_r$ , and $t_s$ are affected by the location of the additional pole. When can the extra pole be ignored?)
% add your discussion here

\subsection{Decomposition of $H_3(s)$\score 14}
\begin{align*}
H_3(s) &= \frac{25}{\left(1+\frac{s}{\alpha \zeta}\right)\left( s^2 + 10 \zeta s + 25\right) } \mbox{~~~($\zeta = 0.6$)} \\
&= \frac{k_1}{1+\frac{5s}{3\alpha}} + \frac{k_2 s}{s^2+6s+25} + \frac{k_3}{s^2+6s+25}
\end{align*}
Using a partial fraction expansion,
\begin{align*}
k_1 &= \mbox{\sc insert expression for $k_1$}\\
k_2 &= \mbox{\sc insert expression for $k_2$}\\
k_3 &= \mbox{\sc insert expression for $k_3$}
\end{align*}
(Discuss how this decomposition helps to explain the effect of the location of an additional pole. In particular, discuss what each term represents. Also discuss $\alpha$'s effect. Which term dominates as $\alpha$ approaches 0? As $\alpha$ approaches $\infty$?)
% add your discussions about the above decomposition
\newline \\[10mm]
\noindent {\bf \large Attachments}
\begin{itemize}
\item Plot from the prelab
\item Plots of overlaid step responses after adding a zero; and after adding a pole
\item Sample figures for calculating $M_p$ , $t_s$ and $t_r$ in Table~\ref{tbl:lab2_q2} and Table~\ref{tbl:lab2_q3}
\end{itemize}

\end{document}
